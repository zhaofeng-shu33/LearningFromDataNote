\documentclass[a4paper, 12pt]{exam}
\usepackage[T1]{fontenc} 
\usepackage{amsmath}
\usepackage{amssymb}
\usepackage{enumerate}
\usepackage{bm}
\newcommand{\bTheta}{\bm{\Theta}}
\newcommand*{\defeq}{\stackrel{\text{def}}{=}}

\usepackage{advdate}
\usepackage{datetime}
\usepackage[mathcal]{eucal}
\usepackage{dsfont}
\usepackage{listings}
\usepackage{xcolor}
\usepackage{graphicx}
\usepackage[noend]{algpseudocode}
\usepackage{algorithmicx,algorithm}
%\usepackage{color} 
\usepackage{listings} 
\usepackage{setspace} 

\definecolor{Code}{rgb}{0,0,0} 
\definecolor{Decorators}{rgb}{0.5,0.5,0.5} 
\definecolor{Numbers}{rgb}{0.5,0,0} 
\definecolor{MatchingBrackets}{rgb}{0.25,0.5,0.5} 
\definecolor{Keywords}{rgb}{0,0,1} 
\definecolor{self}{rgb}{0,0,0} 
\definecolor{Strings}{rgb}{0,0.63,0} 
\definecolor{Comments}{rgb}{0,0.63,1} 
\definecolor{Backquotes}{rgb}{0,0,0} 
\definecolor{Classname}{rgb}{0,0,0} 
\definecolor{FunctionName}{rgb}{0,0,0} 
\definecolor{Operators}{rgb}{0,0,0} 
\definecolor{Background}{rgb}{0.98,0.98,0.98} 
 
\lstdefinelanguage{Python}{ 
numbers=left, 
numberstyle=\footnotesize, 
numbersep=1em, 
xleftmargin=1em, 
framextopmargin=2em, 
framexbottommargin=2em, 
showspaces=false, 
showtabs=false, 
showstringspaces=false, 
frame=l, 
tabsize=4, 
% Basic 
basicstyle=\ttfamily\small\setstretch{1}, 
backgroundcolor=\color{Background}, 
% Comments 
commentstyle=\color{Comments}\slshape, 
% Strings 
stringstyle=\color{Strings}, 
morecomment=[s][\color{Strings}]{"""}{"""}, 
morecomment=[s][\color{Strings}]{'''}{'''}, 
% keywords 
morekeywords={import,from,class,def,for,while,if,is,in,elif,else,not,and,or,print,break,continue,return,True,False,None,access,as,,del,except,exec,finally,global,import,lambda,pass,print,raise,try,assert}, 
keywordstyle={\color{Keywords}\bfseries}, 
% additional keywords 
morekeywords={[2]@invariant,pylab,numpy,np,scipy}, 
keywordstyle={[2]\color{Decorators}\slshape}, 
emph={self}, 
emphstyle={\color{self}\slshape}, 
% 
} 
\usepackage{url}
\newdate{issuedate}{22}{11}{2020}
\newdate{duedate}{06}{12}{2020}

% \newcommand{\duedate}[1][14]{%
% \begingroup
% \AdvanceDate[#1]%
% \today%
% \endgroup
% }%

\usepackage[thehwcnt=3]{iidef}
\thecourseinstitute{Tsinghua-Berkeley Shenzhen Institute}
\thecoursename{Learning From Data}
\theterm{Fall 2020}
\makeatletter
\newcommand{\firstblock}{programming_policies}
\makeatother

\begin{document}
	
	\pagestyle{headandfoot}
	\runningheadrule
	
	
	\newcounter{psctr}
	\setcounter{psctr}{3} % set to the times of problem
	
	\runningheader{Programming Assignment \thepsctr}
	{\textsc{Learning from Data}}
	{ Page \thepage\ of \numpages}
	\firstpagefooter{}{}{}
	\runningfooter{}{}{}
	
	
	\newcounter{Sequ}
	\newenvironment{Sequation}
	{\stepcounter{Sequ}%
		\addtocounter{equation}{-1}%
		\renewcommand\theequation{S\arabic{Sequ}}\equation}
	{\endequation}
	%\topskip0pt
	
	% \vspace*{\fill}
	\centering
	
	% \vspace{0.3em}
	\centering
	\renewcommand{\thequestion}{\arabic{psctr}.\arabic{question}}
	\hwname{Programming Assignment}
	\courseheader
	\begin{flushleft}
		\textbf{Issued:} \displaydate{issuedate} \hfill
		\textbf{Due:} \displaydate{duedate} 
	\end{flushleft}
	
	\hrule 
	
	\input{\firstblock}
	
	%\pointname{}
	%\vspace{\footskip}
	\vspace{1em}
	
	
	%\pointname{}
	%\vspace{\footskip}
	%\vspace{1em}
	
	\begin{questions}

		\question  (\emph{kmeans})
		Recall that kmeans is solving the following optimization
		problem:
		\begin{align*}
		    \min_{C, \mu} \sum_{j=1}^k \sum_{x \in C_j}
		    || x - \mu_j ||^2
		\end{align*}
		Exact solution of the above problem is NP-hard. We proceed
		by an iterative scheme instead. The iterative algorithm
		first initializing the cluster centroids $\mu_1, \dots, \mu_k$. We can choose $k$ different points from $x_1, \dots, x_m$ as the random initialization.
		Then the algorithm iterates between updating the centroids
		(E step) and assigning labels (M step).
		\begin{parts}
		\part[2]
		Please use such algorithm to implement kmeans by completing the code in \textbf{kmeans.py}. Your implementation should follow Algorithm \ref{alg2}.
\begin{algorithm}[H] 
\caption{K-Means Clustering\label{alg2}} %算法的名字
\hspace*{0.02in} {\bf Input:} %算法的输入
data points $x^{(1)}, \dots, x^{(m)}$ and cluster size $k$

\hspace*{0.02in}
{\bf Output: } clustering label vector $y$
\begin{algorithmic}[1]
\State Initialize cluster centroids $\mu_1, \dots, \mu_k \in \mathbb{R}^n$ randomly
\While{not convergent}
\For{$i = 1, \dots, m$}
\State $y_i = \arg\min_j || x^{(i)} - \mu_j||^2$
\EndFor
\For{$j = 1, \dots, k$}
\State $\mu_j = \frac{\sum_{i=1}^m \mathbf{1}\{y_i = j\} x^{(i)}}{\sum_{i=1}^m \mathbf{1}\{y_i = j\} }$
\EndFor
\EndWhile
\end{algorithmic}
\end{algorithm}			
		\part[2] kmeans can be regarded as a special Gaussian mixture model with known uniform variance. If
		the dataset does not follow this assumption, the clustering result may behave poorly and contrary to the
		expectation. In this question, you are required to apply kmeans clustering to an artificial dataset
		with two ellipse contour and present the clustering result in the form of figure. Besides, you should write down
		some analysis in \texttt{README.md} to explain why the
		unexpected clustering result happens. For the experiment code, please
		see \textbf{kmeans-experiment.py} for detail of artificial data generation and result visualization. 
		\end{parts}
		
		\question (\emph{spectral clustering with rbf kernel})
		In this question, we consider unnormalized spectral clustering, which deals with the unnormalized Laplacian matrix of a graph, $L= D - W$. 
		The weighted matrix $W$ is constructed using rbf-kernel,
		\begin{equation}\label{eq:W}
		    W_{ij} = \exp(- \gamma || x^{(i)} - x^{(j)} ||^2)
		\end{equation}
		while the diagonal matrix $D$ is obtained by summing each row of $W$.
		After making dimension reduction from $L$ by choosing its first $k$ eigenvectors $V$
		(corresponding to $k$ smallest eigenvalues), we make clustering
		in the reduced feature space by k-means, which you have already implemented in the previous question.
		\begin{parts}
		\part[4] 
		Please implement spectral clustering by completing the code in \textbf{spectral\_clustering.py}. Your implementation should follow Algorithm \ref{alg1}.
\begin{algorithm}[H] 
\caption{Spectral Clustering\label{alg1}} %算法的名字
\hspace*{0.02in} {\bf Input:} %算法的输入
data points $x^{(1)}, \dots, x^{(n)}$ and cluster size $k$

\hspace*{0.02in}
{\bf Output: } clustering label vector $y$
\begin{algorithmic}[1]
\State Build the similarity matrix $W$ by \eqref{eq:W}
\State Construct unnormalized Laplacian matrix $L$
\State Compute first $k$ eigenvectors $V=[v_1, \dots, v_k]$ of $L$
\State Define $u_i \in \mathbb{R}^k$ as the $i$-th row of $V$,
cluster $u_1, \dots, u_n$ into $k$ clusters using k-means and obtain
the cluster label $y_1, \dots, y_n$
\end{algorithmic}
\end{algorithm}		
		\part[2]
		The performance of spectral clustering is influenced by the scaling parameter $\gamma$. In this question, we use grid search to get optimal $\gamma$. We assume the ground truth label is known in
		advance and use \textsf{adjusted rank index} to evaluate the clustering
		performance for different $\gamma$. \textsf{adjusted rank index}
		is a similarity metric of two label vectors, which gives the highest score 1 if the two underlining clusters are exactly the same.
		Please use this method to tune the parameter $\gamma$ for the given
		three circle dataset. You should report your optimal $\gamma$ and plot the clustering result for this $\gamma$. Besides, write down your analysis in \texttt{README.md} why $\gamma$ in the range you choose
		can produce the desirable result. For the experiment code, please
		see \textbf{spectral-experiment.py} for detail of artificial data generation and result visualization. 
% Please use unnormalized Laplacian for this purpose		    
		\end{parts}
		\question (bonus, 2.5 points)
	    (\emph{spectral clustering with normalized Laplacian})
	    In this question, you are required to implement spectral clustering with
	    normalized Laplacian.
	    Following the convention of \texttt{scipy}, given the weight matrix $W$, the normalized Laplacian is defined mathematically as:
	    \begin{align}
	        W' &= W - \textrm{diag}\{W\} \notag \\
	        D &= \textrm{diag}\{d_1, \dots, d_n\} \textrm{ where } d_i = \sum_{j=1}^n W'_{ij} \notag\\
	        L &= I - D^{-1} W'\label{eq:L}
	    \end{align}
	    Please extend your
	    existing implementation in the class function \texttt{\_get\_embedding} of \textbf{spectral\_clustering.py} to support normalized Laplacian. Your implementation should follow Algorithm \ref{alg3}.
\begin{algorithm}
\caption{Normalized Spectral Clustering\label{alg3}} %算法的名字
\hspace*{0.02in} {\bf Input:} %算法的输入
data points $x^{(1)}, \dots, x^{(n)}$ and cluster size $k$

\hspace*{0.02in}
{\bf Output: } clustering label vector $y$
\begin{algorithmic}[1]
\State Build the similarity matrix $W$ by \eqref{eq:W}
\State Construct normalized Laplacian $L$ from \eqref{eq:L}
\State Compute first $k$ eigenvectors $V=[v_1, \dots, v_k]$ of $L$
\State Define $u_i \in \mathbb{R}^k$ as the $i$-th row of $V$,
cluster $u_1, \dots, u_n$ into $k$ clusters using k-means and obtain
the cluster label $y_1, \dots, y_n$
\end{algorithmic}
\end{algorithm}	
	\end{questions}
	

	\nocite{*}
	\begin{flushleft}
		\textbf{Notice}: \\
		\begin{enumerate}[label=\roman*)]
			\item Use matrix operations other than loops for efficiency. If the running time of Auto-Grading steps exceeds 5 minutes, you will get point deductions.
			\item For algorithm implementation questions, You can \textbf{only} use \texttt{numpy}.
			Using any API of \texttt{scipy} and \texttt{sklearn}
			will lead to point deductions. However,
			for analysis question you are allowed to use any other third-party packages.
			\item For analysis problem, please write down your
			answers in \texttt{README.md}. You can follow \texttt{report-template.md} and click \textbf{Preview changes} on GitHub webpage to view the rendered contents after modification.
			In rare cases when you could not get familiar with the specific syntax
			of markdown, you could upload a \texttt{.docx} or \texttt{pdf} file to the root directory of your GitHub repository and declare where your answers are written in \texttt{README.md}.
		\end{enumerate}
	\end{flushleft}
	
	%\bibliographystyle{plain}
	%\bibliography{ref}
	%\begin{thebibliography}{9}
	%	\bibitem{ridge} \href{https://ncss-wpengine.netdna-ssl.com/wp-content/themes/ncss/pdf/Procedures/NCSS/Ridge_Regression.pdf}{Ridge Regression}
	%	\bibitem{tutorial} \href{https://www.datacamp.com/community/tutorials/tutorial-ridge-lasso-elastic-net}{Regularization: Ridge, Lasso and Elastic Net}
	%\end{thebibliography}
\end{document}
