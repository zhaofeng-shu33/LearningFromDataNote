\documentclass[a4paper, 12pt]{exam}
\usepackage[T1]{fontenc} 
\usepackage{amsmath}
\usepackage{amssymb}
\usepackage{enumerate}
\usepackage{bm}
\newcommand{\bTheta}{\bm{\Theta}}
\newcommand*{\defeq}{\stackrel{\text{def}}{=}}

\usepackage{advdate}
\usepackage{datetime}
\usepackage[mathcal]{eucal}
\usepackage{dsfont}
\usepackage{listings}
\usepackage{xcolor}
\usepackage{graphicx}
\usepackage{url}
\newdate{issuedate}{20}{11}{2020}
\newdate{duedate}{04}{12}{2020}

% \newcommand{\duedate}[1][14]{%
% \begingroup
% \AdvanceDate[#1]%
% \today%
% \endgroup
% }%

\usepackage[thehwcnt=3]{iidef}
\thecourseinstitute{Tsinghua-Berkeley Shenzhen Institute}
\thecoursename{Learning From Data}
\theterm{Fall 2020}
\makeatletter
\newcommand{\firstblock}{programming_policies}
\makeatother

\begin{document}
	
	\pagestyle{headandfoot}
	\runningheadrule
	
	
	\newcounter{psctr}
	\setcounter{psctr}{3} % set to the times of problem
	
	\runningheader{Programming Assignment \thepsctr}
	{\textsc{Learning from Data}}
	{ Page \thepage\ of \numpages}
	\firstpagefooter{}{}{}
	\runningfooter{}{}{}
	
	
	\newcounter{Sequ}
	\newenvironment{Sequation}
	{\stepcounter{Sequ}%
		\addtocounter{equation}{-1}%
		\renewcommand\theequation{S\arabic{Sequ}}\equation}
	{\endequation}
	%\topskip0pt
	
	% \vspace*{\fill}
	\centering
	
	% \vspace{0.3em}
	\centering
	\renewcommand{\thequestion}{\arabic{psctr}.\arabic{question}}
	\hwname{Programming Assignment}
	\courseheader
	\begin{flushleft}
		\textbf{Issued:} \displaydate{issuedate} \hfill
		\textbf{Due:} \displaydate{duedate} 
	\end{flushleft}
	
	\hrule 
	
	\input{\firstblock}
	
	%\pointname{}
	%\vspace{\footskip}
	\vspace{1em}
	
	
	%\pointname{}
	%\vspace{\footskip}
	%\vspace{1em}
	
	\begin{questions}

		\question (2 points) \emph{kmeans}
		Recall that kmeans is solving the following optimization
		problem:
		\begin{align*}
		    \min_{C, \mu} \sum_{j=1}^k \sum_{x \in C_j}
		    || x - \mu_j ||^2
		\end{align*}
		Exact solution of the above problem is NP-hard. We proceed
		by an iterative scheme instead. The iterative algorithm
		first initializing the cluster centroids $\mu_1, \dots, \mu_k$. We can choose $k$ different points from $x_1, \dots, x_n$ as the random initialization.
		Then the algorithm iterates between updating the centroids
		(E step) and assigning labels (M step).
		Please use such algorithm to implement kmeans by completing the code in \textbf{kmeans.py}.
		
		\question (3 points) \emph{spectral clustering}
		In this question, we consider unnormalized spectral clustering, which deals with the unnormalized Laplacian matrix of a graph, $L= D - W$. After making dimension reduction from $L$ by choosing its first $k$ eigenvectors
		(corresponding to $k$ smallest eigenvalues), we make clustering
		in the reduced feature space by k-means, which you have already implemented in the previous question. In this question, 
		please implement spectral clustering by completing the code in \textbf{spectral\_clustering.py}.
		
	\end{questions}
	
	
	\nocite{*}
	\begin{flushleft}
		\textbf{Notice}: \\
		\begin{enumerate}
			\item Use matrix operations other than loops for efficiency. If the running time of Auto-Grading steps exceeds 5 minutes, you will get point deductions.
			\item You are expected to only use \texttt{numpy} packages to implement the algorithms.
			\item All questions assume that the data are not centered around zero. Therefore, you need to consider the bias parameter in your code.
		\end{enumerate}
	\end{flushleft}
	
	%\bibliographystyle{plain}
	%\bibliography{ref}
	%\begin{thebibliography}{9}
	%	\bibitem{ridge} \href{https://ncss-wpengine.netdna-ssl.com/wp-content/themes/ncss/pdf/Procedures/NCSS/Ridge_Regression.pdf}{Ridge Regression}
	%	\bibitem{tutorial} \href{https://www.datacamp.com/community/tutorials/tutorial-ridge-lasso-elastic-net}{Regularization: Ridge, Lasso and Elastic Net}
	%\end{thebibliography}
\end{document}